\chapter*{Bevezetés}
\addcontentsline{toc}{section}{Bevezetés}
A szakdolgozatom témája adatbázisok logikai relációs adatmodell szerinti példák generálása és megoldása, illetve egy gyakorló plugin elkészítése Moodle\footnote{Link:\href{https://moodle.org}{https://moodle.org}} rendszerben. Ezt PHP nyelven implementálom és építem be. A Moodle egy ingyenes és nyílt forráskódú oktatási tananyagkezelő rendszer, mely PHP-ben íródott és bárki számára használható és fejleszthető. Ebbe tudunk generálni saját oldalakat, készíthetünk saját adatbázist, tölthetünk fel fájlt, építhetünk bele saját weblapot vagy plugin-t, vagy akár más nyelvet is implementálhatunk bele, például JavaScript kódot. Emiatt elég elterjedt lett a különböző középiskolákban, illetve egyetemeken. Ráadásul, mivel egy webszerverről beszélünk, a cross-platform lehetősége is adott, sőt készítve lett egy Moodle applikáció Androidra és IOs-re is. Így összességében egy széles körben használható rendszer. \hfill \newline
A szakdolgozatban található adatbázissémákat 4 fajtára bontottam szét, amelyeket az alábbiak:\hfill \newline
\begin{itemize}
    \item Egyszerű: A relációs séma 7-9 attribútumot és 3 függést tartalmaz, melyek egyértelműen meghatározzák a séma \textbf{2NF}, \textbf{3NF} és \textbf{BCNF} alakját. Az első függés mindig egy részhalmaza a kulcsnak, amely egy nemkulcselemek egy részhalmazára mutat, a második az első függés jobboldali elemeinek egy részhalmazából egy még nem érintett nemkulcsbeli elemek egy részhalmazára mutat, illetve a harmadik függésben a teljes kulcs a maradék nemkulcsbeli elemekre mutat.
    \item Összetett: A relációs séma 7-8 attribútumot és 3 függést tartalmaz, viszont itt a függések baloldalai tartalmazhatnak nemkulcsbeli és kulcsbeli elemek párosításait is, illetve jobban oda kell figyelni a zártságra.
    \item Dupla: A relációs séma 9 attribútumot és 4 függést tartalmaz, melyekből 2 darab a \textbf{2NF}-et, 2 darab pedig a \textbf{3NF}-et sérti.
    \item Zártság: A relációs séma 7-8 attribútumot és 4 függést tartalmaz, viszont a séma \textbf{2NF} és \textbf{3NF} meghatározásához oda kell figyelni a sémát sértő függés bal oldalán lévő attribútumhalmaz zártságára, hogy ténylegesen megfeleljen.
\end{itemize}
Ezeket a példafeladatokat egy általános megoldó \textbf{2NF}, \textbf{3NF} és \textbf{BCNF} megoldóval értékelem ki és együttesen implementálom egy saját készítésű Moodle pluginben.\hfill \newline

Ezen algoritmusok elsajátítása fontos egy informatikus hallgató számára, mivel az adatok többszörös tárolását, redundancia elkerülését, ezáltal hatékonyabb adatbázis-struk-túra kialakítását lesznek képesek elsajátítani. Található az interneten több megoldó is, viszont ezek esetében nekünk kell megadni a feladatot, nincs benne generátor. Ezenkívül ilyen Moodle plugin jelenleg nem található a sablonok között, így ez jó lehet arra is, ha valaki ezt tovább akarja fejleszteni. Végezetül pedig készült egy másik, hasonló plugin is, amely egy normalizálási feladat diák által adott megoldását képes összehasonlítani a valódi megoldással és kiértékelni azt, így ezzel a pluginnel kombinálva egy tesztsegédprogram jöhet létre, melyet gyakorlásra és tesztírásra is lehet használni.
