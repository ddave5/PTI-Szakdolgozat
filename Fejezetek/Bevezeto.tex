\chapter*{Bevezetés}
\addcontentsline{toc}{section}{Bevezetés}
A szakdolgozatom témája logikai adatbázisok relációs adatmodell szerinti példák generálása és megoldása, illetve egy gyakorló plug-in elkészítése Moodle rendszerben. Ezt PHP nyelven implementálom és építem be. A moodle egy ingyenes és nyílt forráskódú oktatási webszerver, mely PHP-ben íródott és bárki számára használható és fejleszthető. Ebbe tudunk generálni saját oldalakat, készíthetünk saját adatbázist, tölthetünk fel file-t, építhetünk bele saját weblapot vagy plug-in-t, vagy akár más nyelvet is implementálhatunk bele, például JavaScript kódot. Emiatt elég elterjedt lett a különböző közép-iskolákban, illetve egyetemeken. Ráadásul, mivel egy webszerverről beszélünk, a cross-platform lehetősége is adott, sőt készítve lett egy moodle applikáció Androidra és IOs-re is. Így összességében egy széles körben elterjedt rendszer. \hfill \newline
A szakdolgozatban található adatbázissémákat 4 fajtára bontottam szét, amelyeket az alábbiak:\hfill \newline
\begin{itemize}
    \item Egyszerű: A relációs séma 6-8 attribútumot és 3 függést tartalmaz, melyek egyértelműen meghatározzák a séma 2NF,3NF és BCNF alakját. Az első függés mindig egy részhalmaza a kulcsnak, amely egy nemkulcselemek egy részhalmazára mutat, a második az első függés jobboldali elemeinek egy részhalmazából egy még nem érintett nemkulcsbeli elemek egy részhalmazára mutat, illetve a harmadik függésben a teljes kulcs a maradék nemkulcsbeli elemekre mutat.
    \item Összetett: A relációs séma 7-8 attribútumot és 3 függést tartalmaz, viszont itt a függések baloldalai tartalmazhatnak nemkulcsbeli és kulcsbeli elemek párosításait is, illetve jobban oda kell figyelni a zártságra.
    \item Dupla: A relációs séma 9 attribútumot és 4 függést tartalmaz, melyekből 2 darab a 2NF-et, 2 darab pedig a 3NF-et sérti.
    \item Zártság: A relációs séma 7-8 attribútumot és 4 függést tartalmaz, viszont a séma 2NF és 3NF meghatározásához oda kell figyelni a sémát sértő függés bal oldalán lévő attribútumhalmaz zártságára, hogy ténylegesen megfeleljen.
\end{itemize}
Ezeket a példafeladatokat egy általános megoldó 2NF, 3NF és BCNF solverrel értékelem ki és együttesen implementálom egy saját készítésű Moodle plug-inben.