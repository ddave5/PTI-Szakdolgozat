\chapter{A feladatok generálása}

Összegezve a feladatom relációsémák generálása és ezekre válasz bekérése felhasználótól, majd ezek mentése Moodle plugin formában volt. Ezeket a plugin teljes mértékben végre tudja hajtani, odafigyelve, hogy az aktuális felhasználónak milyen rangja van. A Moodle összetettségéből kifolyólag sokkal több mindent lehetne implementálni, de azok már olyan funkciók lennének, amellyel extra feladatok ellátását képezné. Mivel a feladatkiírás nem tartalmazza, hogy ezen feladatokat értékeljük is ki, így ez a plugin önmagában nem képes tesztelési szinten működni, egyesével és külön külön kellene egy tanárnak kiértékelnie, viszont egy dolgozatra való felkészüléshez ideális tud lenni. Emiatt lett benne több féle nehézség, hogy mindenki be tudja gyakorolni a saját szintjéhez megfelelően ezeket az algoritmusokat. \par
Saját véleményem szerint a plugin egy kompakt, egyszerű kialakítású, de annál összetettebb program lett, amely a későbbiekben néhány fejlesztés után az egyetemi oktatásban is megállhatja a helyét, mint gyakorló és tesztelőprogram. 