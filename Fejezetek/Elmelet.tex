\chapter{Általános definíciók}
\section{Adatbázis, adatmodell}

\begin{defi}[Adat]
Az \textbf{adat} az elemi ismeret. Olyan számok, karakterek sorozata, melyhez nincs jelentés tulajdonítva. Onnantól, hogy tudjuk, mit jelent beszélhetünk \textbf{információ}ról. 
\end{defi}

Az adatokat szeretnénk értelmezni, könnyen kezelhető formába helyezni akár a programon kívül tároltatni. Erre szolgálnak az adatbázisok.

\begin{defi}[Adatbázis, sor, rekord]
Az \textbf{adatbázis} adott formátum és rendszer szerint tárolt adatok együttesét jelenti. Ennek alapvető adategységét \textbf{sor}nak vagy \textbf{rekord}nak hívjuk. Sor esetén egy teljes adatbázisbeli objektum értékét, míg rekord esetén egy objektum adott tulajdonságát értjük.
\end{defi}

\begin{defi}[Egyed,tulajdonság,kapcsolat]
\textbf{Egyednek} nevezzük azt az objektumot, amelyről szeretnénk adatokat tárolni. Ezeknek az egyedeknek egy-egy jellemzőjét \textbf{attribútum}nak hívjuk, illetve az egyedek között \textbf{kapcsolat}ok alakulhatnak ki, melyek meghatározhatják a különböző egyedek egyes tulajdonságait.
\end{defi}

Az adatokat szeretnénk rendszerezni is az adatbázisban, erre több módszer is szolgál, amelyek évek-évtizedek alatt alakultak ki. Ezen modellek a következők:
\begin{itemize}
    \item Hiearchikus modell
    \item Hálós modell
    \item Relációs modell
    \item Objektumorientált modell
    \item Objektum-relációs modell
\end{itemize}
A szakdolgozat a relációs modell-lel foglalkozik, amely a mai napig széles körben használt modell.

\begin{defi}[Relációs adatmodell]
A \textbf{relációs adatmodell}ben az adatokat és a köztük lévő kapcsolatokat kétdimenziós táblákban tárolja. A táblákban az azonos sorban álló egyedek alkotnak egy relációt. Az erre a modellre épülő adatbáziskezelő-rendszereket RDBMS-nek (Relational Database Management System) nevezzük.
\end{defi}

\begin{defi}[Relációs adatmodell attribútuma, attributumhalmaza, relációséma]
A relációs adatmodellben \textbf{attribútum}nak egy névvel és értéktartománnyal ellátott tulajdonságot nevezünk. A $Z$ attribútum értéktartományát $dom(Z)$-vel fogjuk jelölni, az angol "domain" szóból rövidítve. Egy relációs adatbázis attribútumainak összessége az \textbf{attribútumhalmaz}. Az attribútumhalmaz egy névvel ellátva pedig a \textbf{relációséma}. Legyen $A = \{A_1,\ldots,A_n\}$ egy attribútumhalmaz, és legyen a séma neve $R$, ekkor az a relációsséma jele $R(A_1,\ldots,A_n)$ rövidítve $R(A)$. Ha két séma, legyen $R$ és $S$ azonos attribútumokat is tartalmaz, akkor azt jelölhetjük $R.A_i$ és $S.A_i$-vel.
\end{defi}

\begin{defi}[$R(A)$ feletti $T$ reláció]
Az \textbf{$R(A)$ feletti $T$ reláció} alatt egy $T \subseteq dom(A_1) \times \ldots \times dom(A_n)$ halmazt értünk.  
\end{defi}

Viszont szeretnénk egy adattábla soraira egyértelműen tudjunk hivatkozni, ahhoz kell egy olyan attribútum, amely minden sorban különbözik. Erre fog szolgálni nekünk a kulcs.

\begin{defi}[Kulcs, szuperkulcs, egyszerű és összetett kulcs, elsődleges kulcs, külső kulcs]
A \textbf{szuperkulcs} az adattábla azon attribútumainak különleges tulajdonsága, mely egyértelműen meghatározza a tábla sorait. Formálisan  $R(A)$ sémában a $K \subseteq A$ halmaz szuperkulcs, ha bármely $R$ feletti $T$ tábla bármely két sora különbözik $K$-n. A szuperkulcs lehet a teljes sor is, de ebben az esetben minden egyes sorazonosítás esetén minden sort külön-külön meg kellene vizsgálni, amely nem praktikus. Emiatt célunk megkeresni azt a szuperkulcsot, mely a legkevesebb elemből áll. A \textbf{kulcs} az a szuperkulcs, melynek az elemszáma minimális, azaz egyetlen részhalmaza sem lesz szuperkulcs. Ha a kulcs egyelemű, akkor \textbf{egyszerű kulcs}ról, ha pedig többelemű, akkor \textbf{összetett kulcs}ról beszélünk. Az is megtörténhet, hogy egy adatbázisnak több kulcsa is van, ekkor érdemes kiválasztani egyet ezek közül és ez lesz az \textbf{elsődleges kulcs}. Ezenkívül pedig ha egy $R(A)$ egy $K\subseteq A$ részhalmaza egy másik relációs sémának az elsődleges kulcsaira mutat, akkor a $K$ \textbf{külső kulcs}.
\end{defi}

\begin{pld}
Vegyünk egy hétköznapi példát egy relációs adatbázis-modellre:
\begin{center}
    Könyv(\underline{ISBN}, \textit{kiadókód}, író, cím, megjelenés éve) \break
    Kiadó(\underline{kiadókód}, kiadó neve, város, cím)
\end{center}
Ebben a példában az ISBN egy attribútum, az $\{\text{ISBN, kiadókód, író, cím, megjelenés éve} \}$ egy attribútumhalmaz, Könyv a séma neve, $T = dom(\text{ISBN}) \times dom(\text{kiadókód})$ egy Könyv feletti $T$ reláció.
Szuperkulcs lehet például a $\{$ISBN, kiadókód, író, cím, megjelenés éve $\}$ vagy $\{$ISBN, kiadókód$\}$ vagy $\{$ ISBN $\}$. A kulcsa a Könyv relációsémának az ISBN (ezt mindig aláhúzással jelöljük), a Kiadónak pedig a kiadókód. Illetve a Könyv kiadókód attribútuma lesz egy külső kulcs.
\end{pld}

\section{Normálforma}

Ha minden adatot egy relációssémában tárolunk, az idő és tárigényes is lehet, mivel sokkal körülményesebb lehet a módosítás, törlés illetve a beszúrás. Emiatt egy nagyobb Relációsémát érdemesebb több, kisebb relációsémára bontani, amelyek kapcsolatban állnak egymással. Így időt és tárigényt csökkenthetünk, sőt az adattábláink is könnyebben átláthatóvá válnak. Erre szolgál a normalizálás. Viszont mielőtt ezt meg tudjuk tenni, elsőnek vezessünk be néhány alapdefiníciót.

\begin{defi} [Funkcionális függőség]
Legyen $R(A_1,\ldots,A_n)$ egy relációséma és $P,Q \subseteq \{A_1,\ldots,A_n\}$. Azt mondjuk, hogy $P$-től \textbf{funkcionálisan függ} $Q$ $(P \rightarrow Q)$, ha bármely $R$ feletti $T$ tábla esetén valahányszor két sor megegyezik $P$-n, mindannyiszor megegyezik $Q$-n is, azaz 
$\forall t_i, t_j \in T$ esetén $ t_i(P) = t_j(P) \Rightarrow t_i(Q) = t_j(Q)$. 
\end{defi}

\begin{defi}[Triviális/nemtriviális függés, teljesen nemtriviális függés]
Legyen $R(A_1,\ldots,A_n)$ egy relációséma és $P,Q \subseteq \{A_1,\ldots,A_n\}$. A $P \longrightarrow Q$ függést \textbf{triviális}nak nevezzük, ha $Q \subseteq P$, ellenkező esetben \textbf{nemtriviálisnak}. A $P \longrightarrow Q$ függést \textbf{teljesen nemtriviálisnak} nevezzük, ha $P \cap Q = \emptyset$.
\end{defi}

\begin{pld}
Vegyük az alábbi relációsémát:
\begin{center}
    Felhasználó(felhasználónév, e-mail, jelszó, születési év)
\end{center}
Ebben az alábbi függések a nemtriviálisak:
\begin{itemize}
    \item $\{\text{felhasználónév}\}\longrightarrow \{\text{e-mail}\}$
    \item $\{\text{felhasználónév}\}\longrightarrow \{\text{jelszó}\}$
    \item $\{\text{e-mail}\}\longrightarrow \{\text{felhasználónév}\}$
    \item $\{\text{e-mail}\}\longrightarrow \{\text{jelszó}\}$
    \item $\{\text{felhasználónév}\}\longrightarrow \{\text{e-mail,jelszó,születési év}\}$
\end{itemize}
Továbbá az alábbi függések teljesen nemtriviálisak:
\begin{itemize}
    \item $\{\text{felhasználónév,e-mail}\}\longrightarrow \{\text{jelszó}\}$
    \item $\{\text{felhasználónév, születési év}\}\longrightarrow \{\text{e-mail}\}$
    \item $\{\text{e-mail, jelszó}\}\longrightarrow \{\text{felhasználónév}\}$
\end{itemize}
\end{pld}

\begin{defi}[Attribútumhalmaz lezártja]
Egy $X$ \textbf{attribútumhalmaz lezártja} az $F$ függőségi halmaz szerint $X^+ = \{ A_i | X \longrightarrow A_i\}$ halmaz, tehát azon $A_i$ attribútumokból áll, melyekre az $X \longrightarrow A_i$ függőség $F$-ből levezethető.
\end{defi}

\begin{tet}[Attribútumhalmaz lezártjának kiszámítási módja] \label{closedAttribute}
Az alábbi algoritmus megadja egy $X$ attribútumhalmaz lezártját $F$ felett.
\begin{enumerate}
    \item Legyen $X^{(0)} = \{X\}$. Legyen $i = 0$.
    \item Keressünk egy $(P \longrightarrow Q) \in F$ függőséget úgy, hogy $P \subseteq X^{(i)}$ és $Q \not\subseteq X^{(i)}$. Ha nem találunk ilyet, akkor $X^+ = X^{(i)}$ és vége.
    \item Legyen $i = i + 1$ és $X_i = X_i \cup Q$, majd ugorjunk a $2.$ lépésre.
\end{enumerate}
\end{tet}

\begin{pld}
Vegyük az alábbi relációsémát a függőségeivel:
\begin{center}
    $R(a,b,c,d,e,f,g)$ \break
    $\{a\} \longrightarrow \{e,f\}$\break
    $\{f\} \longrightarrow \{g\}$\break
    $\{b\} \longrightarrow \{d\}$  
\end{center}
Ekkor $\{a,b\}$ lezártja:
\begin{itemize}
    \item $X^{(0)} = \{a,b\}.$
    \item $X^{(1)} = \{a,e,f\}$, mivel $\{a\} \longrightarrow \{e,f\}$ függőséggel tudjuk bővíteni.
    \item $X^{(2)} = \{a,e,f,g\}$, mivel $\{f\} \longrightarrow \{g\}$ függőséggel tudjuk bővíteni.
    \item $X^{(3)} = \{a,d,e,f,g\}$, mivel $\{b\} \longrightarrow \{d\}$ függőséggel tudjuk bővíteni.
    \item $X^{(4)} = \{a,d,e,f,g\} = X^{(3)}$ és az algoritmusnak vége.
\end{itemize}
\end{pld}

\begin{defi}[implied/nonimplied extraneous attribute]
Legyen $R(A)$ egy relációséma és $F$ függőségi halmaz, illetve $X,Y \subseteq A$ részhalmazok. Ekkor azt mondjuk, hogy egy $A_i$ \textbf{extraneous attribútum} egy $X \longrightarrow Y$ $F$ feletti függőségben, ha $A \in (A-A_i)^+$, Ha $A_i \in (A-A_i)^+$, akkor $A_i$ \textbf{implied extraneous attribútum}, viszont ha csak extraneous, de nem implied, akkor \textbf{nonimplied extraneous attribútum}.
\end{defi}

\begin{pld}
Vegyük $R(a,b,c)$ relációsémát és $\{a,b\} \longrightarrow \{c\}, \{a\} \longrightarrow \{c\}$ függőségeket. Ekkor $b$ nonimplied extraneous attribútum a $\{a,b\} \longrightarrow \{c\}$ függőségben. \hfill \break
Illetve vegyünk egy másik függőségi halmazt, melynek elemei $\{a,b\} \longrightarrow \{c\}, \{a\} \longrightarrow \{b\}$. Ekkor $b$ implied extraneous attribútum a $\{a,b\} \longrightarrow \{c\}$ függőségben.
\end{pld}

\begin{defi}[Fölösleges függőség]
Legyen $R(A)$ relációséma és $F$ függőségi halmaz, illetve $X,Y,Z \subseteq A$ részhalmazok. Ekkor azt mondjuk, hogy $X \longrightarrow Z$ \textbf{fölösleges függőség} $F$-ben, ha létezik egy olyan $Y \longrightarrow Z$, amely segítségével $X \longrightarrow Y \in F^+$.
\end{defi}

\begin{defi}[Teljes és részleges függőség]
Legyen $R(A)$ relációséma és $F$ függőségi halmaz, illetve $X,Y \subseteq A$ részhalmazok. Azt mondjuk, hogy $X \longrightarrow Y$ \textbf{teljes függőség}, ha bármely $A_i \in X$ elhagyása esetén az $X \longrightarrow Y$ már nem teljesülne. Ezenkívül $X \longrightarrow Y$ \textbf{részleges függőség}, ha $Y$ nem tartalmaz kulcs attribútumot és $B \in X$ elhagyása esetén a függés még mindig teljesül.
\end{defi}

\begin{defi}[Elsődleges és másodlagos attribútum]
Egy attribútumot \textbf{elsődleges attribútum}nak nevezünk, ha szerepel a relációséma valamely kulcsában, ellenkező esetben \textbf{másodlagos attribútum}nak hívjuk.
\end{defi}

\begin{defi}[Közvetlen és tranzitív függőség]
Legyen $X,Z \subseteq A$ úgy, hogy $X \rightarrow Z$. Azt mondjuk, hogy $X \longrightarrow Z$ \textbf{tranzitív függés}, ha létezik olyan $Y \subseteq A$, amelyre $X \longrightarrow Y$ és $Y \longrightarrow Z$ úgy, hogy az $Y \longrightarrow Z$ teljesen nemtriviális és $X$ nem függ $Y$-tól. Ellenkező esetben azt mondjuk, hogy $X \longrightarrow Z$ \textbf{közvetlen függés}.
\end{defi}

\begin{defi}[Normálformák]
Egy relációséma normálformái az alábbiak:
\begin{itemize}
    \item Egy relációséma \textbf{1NF}-ben van, ha az attribútumok értéktartománya csak egyszerű adatokból áll.
    \item Egy relációséma \nfk-ben van, ha minden másodlagos attribútum teljesen függ bármely kulcstól.
    \item Egy relációséma \nfh-ben van, ha minden másodlagos attribútum közvetlenül függ bármely kulcstól, azaz nincs kulcstól vett tranzitív függés.
    \item Egy relációséma \textbf{Boyce-Codd normálformá}-ban van (röviden \BCNF-ben), ha bármely nemtriviális $L \longrightarrow B$ függés esetén $L$ szuperkulcs.
\end{itemize}
\end{defi}

\begin{megj}
A szakdolgozat \nfk, \nfh és \BCNF normalizálással foglalkozik, de ezenkívül még létezik \textbf{4NF}, \textbf{ETNF}, \textbf{5NF}, \textbf{DKNF} és \textbf{6NF} normálformák is, viszont ezekre itt nem térek ki.
\end{megj}

\begin{tet}[A normálformák közti összefüggés]
Ha egy relációséma \BCNF-ben van, akkor \nfh-ben is, ha \nfh-ben van, akkor \nfk -ben is, illetve ha \nfk-ben van, akkor \textbf{1NF}-ben is. 
\end{tet}

\begin{pld}
Vegyük az alábbi relációsémát:
\begin{center}
    Reptér(\underline{név},\underline{város},ország, napi járatszám, főváros)
\end{center}
Illetve feleltessünk meg hozzá függőségeket is:
\begin{center}
    $\{\text{név,város}\} \longrightarrow \{\text{napi járatszám}\}$ \break
    $\{\text{város}\} \longrightarrow \{\text{ország, főváros}\}$ \break
    $\{\text{ország}\} \longrightarrow \{\text{főváros}\}$ 
\end{center}
Mivel minden attribútum elemi attribútumokból áll, ezért a relációs modell \textbf{1NF}-ben van. Viszont mivel egy város nevéből meg tudjuk hatázoni, hogy melyik országban van és mi a fővárosa, így nem minden másodlagos attribútum függ bármely kulcstól, tehát a $\{\text{város}\} \rightarrow \{\text{ország, főváros}\}$ függés sérti a \nfk-et. Emiatt ez a függőség alapján szét lehet bontani a relációs modellt az alábbira:
\begin{center}
    Reptér(\underline{név},\underline{város},napi járatszám) \break
    Helyezkedés(\textit{város}, ország, főváros)
\end{center}
Ha jobban megfigyeljük, látszik, hogy a városból adódik az ország és a főváros, de az országból meg tudjuk mondani annak fővárosát, így találhatunk ebben a modellben kulcstól vett tranzitív függést. Így a példánk tovább bontani a $\{\text{ország}\} \rightarrow \{\text{főváros}\}$ függés alapján, ami így néz ki:
\begin{center}
    Reptér(\underline{név},\underline{város},napi járatszám) \break
    Helyezkedés(\textit{város}, \textit{ország})\break
    Székhely(\underline{ország}, főváros)
\end{center}
és így teljesíti a \nfh-et is. Végezetül láthatjuk, hogy minden egyes másodlagos attribútum nemtriviális szuperkulcsbeli elemtől függ, emiatt a feladat már teljesíti a \BCNF-et.\hfill \break
Összegezve az példánk \BCNF alakja:
\begin{center}
    Reptér(\underline{név},\underline{város},napi járatszám) \break
    Helyezkedés(\textit{város}, \textit{ország})\break
    Székhely(\underline{ország}, főváros)
\end{center}
\end{pld}
