%% Az itrodalomjegyzek keszitheto a BibTeX segedprogrammal:
%\bibliography{diploma}
%\bibliographystyle{plain}

%VAGY "kézzel" a következõ módon:

\begin{thebibliography}{9}
%10-nél kevesebb hivatkozás esetén

%\begin{thebibliography}{99}
% 10-nél több hivatkozás esetén

\addcontentsline{toc}{section}{Irodalomjegyzék}

%Elso szerzok vezetekneve alapjan ábécérendben rendezve.


%folyóirat cikk: szerzok(k), a folyóirat neve kiemelve,
%az evfolyam felkoveren, zarojelben az evszam, vegul az oldalszamok es pont.
\bibitem{Gischer}
J. L. Gischer,
The equational theory of pomsets.
\emph{Theoret. Comput. Sci.}, \textbf{61}(1988), 199--224.

%könyv (szerzo(k), a könyv neve kiemelve, utana a kiado, a kiado szekhelye, az evszam es pont.)
\bibitem{Pin}
J.-E. Pin,
\emph{Varieties of Formal Languages},
Plenum Publishing Corp., New York, 1986.

\end{thebibliography}