%A \chapter* parancs nem ad a fejezetnek sorszámot
\chapter*{Feladatkiírás}
%A tartalomjegyzékben mégis szerepeltetni kell, mint szakasz(section) szerepeljen:
\addcontentsline{toc}{section}{Feladatkiírás}

Feladatgeneráló program készítése relációs adatmodell sémáinak normalizálásához

Az Adatbázisok kurzuson fontos hangsúlyt kap az relációs adatbázissémák normalizálása. A kurzus keretén belül, ebből a témakörből is zárthelyi dolgozatot írnak a hallgatók. Jelen szakdolgozati téma ezt hivatott segíteni.

A jelentkező feladata olyan Moodle plugin modul írása, amely az alábbi funkciókat látja el:
\begin{itemize}
    \item Feladatgenerálás különböző nehézségi szinten:
    \begin{itemize}
        \item relációs adatbázisséma és funkcionális függőséghalmaz generálása a feladatokhoz
    \end{itemize}
    \item A generált feladatok normalizálása 2. normálformába, 3. normálformába és Boyce-Codd normálformába.
    \item A feladatokat a Moodle felületén meg kell oldania a tanulónak, be kell tudni adni a megoldást, amelyet a rendszernek el kell tárolnia.
    \item A tanárnak látnia kell a beküldött feladatokat.
\end{itemize}
A hallgatói megoldások kiértékelését nem kell megvalósítani, azt egy másik program végzi.
