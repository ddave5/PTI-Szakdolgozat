%A \chapter* parancs nem ad a fejezetnek sorszámot
\chapter*{Feladatkiírás}
%A tartalomjegyzékben mégis szerepeltetni kell, mint szakasz(section) szerepeljen:
\addcontentsline{toc}{section}{Feladatkiírás}

\begin{itemize}
    \item A témát meghirdető oktató neve: Dr. Németh Gábor.
    \item A témát meghirdető tanszék neve: Képfeldolgozás és Számítógépes Grafika Tanszék.
    \item A téma címe: Feladatgeneráló program készítése relációs adatmodell sémáinak normalizálásához.
    \item A téma angol címe: Excercise generator for relational schema normalization.
    \item Típus: Szakdolgozat.
    \item A feladat rövid leírása: 
    Feladatgeneráló program készítése relációs adatmodell sémáinak normalizálásához

    Az Adatbázisok kurzuson fontos hangsúlyt kap az relációs adatbázissémák normalizálása. A kurzus keretén belül, ebből a témakörből is zárthelyi dolgozatot írnak a hallgatók. Jelen szakdolgozati téma ezt hivatott segíteni.
    
    A jelentkező feladata olyan Moodle plugin modul írása, amely az alábbi funkciókat látja el:
    \begin{itemize}
        \item Feladatgenerálás különböző nehézségi szinten:
        \begin{itemize}
            \item relációs adatbázisséma és funkcionális függőséghalmaz generálása a feladatokhoz
        \end{itemize}
        \item A generált feladatok normalizálása 2. normálformába, 3. normálformába és Boyce-Codd normálformába.
        \item A feladatokat a Moodle felületén meg kell oldania a tanulónak, be kell tudni adni a megoldást, amelyet a rendszernek el kell tárolnia.
        \item A tanárnak látnia kell a beküldött feladatokat.
    \end{itemize}
    A hallgatói megoldások kiértékelését nem kell megvalósítani, azt egy másik program végzi.
    \item Milyen szakos hallgató jelentkezhet: 1 fő, Programtervező informatikus BSc szakos hallgató.
    \item Szakirodalom: angol és magyar nyelv.
    \item Előismeretek: nem szükségesek.

\end{itemize}
