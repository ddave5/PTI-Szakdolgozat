\chapter{A feladatok megoldása}

\section{Az általános megoldás problémái}

A generált feladatoknál látható, hogy nehéz mondani egy nagy, egységes megoldási módot. Pontosabban \nfh-re és \BCNF-re még lehet is megadni egy algoritmust. \textbf{3NF} esetén: \par
Legyen egy $R(A)$ relációsémánk és tegyük fel, hogy teljesíti az \textbf{1NF}-et, illetve legyen $F$ egy $R(A)$-hoz tartozó függéshalmaz.
\hfill \break
\begin{algorithmic}
    \Function{NF3}{$R$}
        \State $R2nf \gets NF2(R)$
        \State $R3nf \gets []$
        \ForAll{$relaciosema(A) \in R2nf$} 
            \ForAll{$ U \rightarrow V, X \rightarrow Y \in F_{relaciosema}$}
                \If{$ U \in Key_{relaciosema} \text{ és }  X,Y \in V$}
                    \State $R3nf \gets R(X\cup Y)$
                    \State $relaciosema(A) \gets relaciosema(A \backslash ((X\cup Y) \backslash Key_{relaciosema}))$ 
                    \State Remove useless dependencies from $R(X \cup Y)$
                \EndIf
            \EndFor
            \State Remove useless dependencies from $relaciosema(A)$
            \State $R3nf \gets relaciosema(A)$
        \EndFor
    \EndFunction
\end{algorithmic}
\hfill \break
Az algoritmus egyszerűen végignézi a relációséma függéseit és ahol talál tranzitív függést, ott létrehoz egy új relációsémát, illetve módosítja az eredetit. Ennek a pszeudokódnak a logikai alapjait tanuljuk Adatbázisok gyakorlaton is, mivel egy egyszerű és könnyen átlátható módszer egy adatbázis \nfh-ra hozáshoz. \par
\textbf{BCNF} esetén sincs nehezebb dolgunk:
Legyen egy $R(A)$ relációsémánk és tegyük fel, hogy teljesíti az \textbf{1NF}-et, illetve legyen $F$ egy $R(A)$-hoz tartozó függéshalmaz.
\hfill \break
\begin{algorithmic}
    \Function{BCNF}{$R$}
        \State $R3nf \gets NF3(R)$
        \State $BCNF \gets []$
        \ForAll{$relaciosema(A) \in R3nf$} 
            \ForAll{$ X \rightarrow Y \in F_{relaciosema}$}
                \If{$ X \subsetneq Key_{relaciosema}$}
                    \State $BCNF \gets R(X\cup Y)$
                    \State $relaciosema(A) \gets relaciosema(A \backslash ((X\cup Y) \backslash Key_{relaciosema}))$ 
                    \State Remove useless dependencies from $R(X \cup Y)$
                \EndIf
            \EndFor
            \State Remove useless dependencies from $relaciosema(A)$
            \State $BCNF \gets relaciosema(A)$
        \EndFor
    \EndFunction
\end{algorithmic}
\hfill \break
Ez a pszeudokód pedig végignézi az összes relációsémához tartozó függések bal oldalát, és ha az nem szuperkulcs, akkor felbontja az által és így hoz létre egy új relációsémát. \par
A gond leginkább a \nfk-nál fog történni. Az egyszerű feladat esetén meg kell keresni egyszer a részkulcs közvetlen függéseit, dupla esetén többször is ellenőrizni kell, zártság esetén pedig nem elég a közvetlen függéseket ellenőrizni, hanem teljesen a részkulcs zárt részhalmazát kell venni. Emiatt érdemes egy olyan függvényt létrehozni, amely átlalánosságban készíti el a relációs modell \textbf{2NF} alakját. Viszont az alábbi tétel segíthet nekünk ebben

\begin{tet}
Legyen $R$ egy \textbf{1NF}-ben lévő relációséma, $F$ a sémához tartozó függéshalmaz és $F_p$ pedig a részleges függéseinek halmaza. Ekkor
\begin{itemize}
    \item Ha $F_p = \emptyset$, akkor $R$ \nfk-ben van.
    \item Ha $F_p \not = \emptyset$, akkor $R$ nincs \nfk-ben.
\end{itemize}
\end{tet}

Így ez lesz az irány, amely alapján fogjuk megcsinálni a \nfk-t.

\section{\nfk-re hozó algoritmus}

Hogy általánosságban működhessen az algoritmus, elsőnek a felesleges változókat és függéseket szüntessük meg. Ehhez az alábbi függvényeket használjuk:
\hfill \break
\begin{algorithmic}
    \Function{createAttributePlus}{$R$, $attributumhalmaz$}
        \State $closed \gets attributumhalmaz$
        \While{ $X \rightarrow Y \in F$ exists}
            \If{$X \subseteq closed \text{ és } Y \not \subset closed$}
                \State $closed \gets closed \cup Y$
            \EndIf
        \EndWhile
        \State \textbf{return} $closed$;
    \EndFunction
    \hfill \break
    
    \Function{generateRSandLSandLR}{$R$}
    \State $RS \gets []$
    \State $LS \gets []$
    \State $LR \gets []$
    \ForAll{ $X \rightarrow Y \in F$}
        \State $LS \gets X$
        \State $LR \gets Y$
        \ForAll{$a_i \in A$}
            \If{$a_i \in LS \text{ és } a_i \in RS$}
                \State $LS \gets LS \backslash a_i$
                \State $LR \gets LR \backslash a_i$
                \State $LR \gets a_i$
            \EndIf
        \EndFor
    \EndFor
    \State $unique(LS),unique(RS),unique(LR)$
    \State \textbf{return} $(RS,LS,LR)$
    \EndFunction
    \hfill \break
    
    \Function{removeExtraneousAttributes}{$R$}
    \State $G \gets F$
    \ForAll{$X \rightarrow Y \in G$}
        \If{ $|X| \geq 1$}
            \State $Z \gets X$
            \ForAll{$z \in Z$}
                \If{$z \in LR$}
                    \State $H \gets G - (X \rightarrow Y) \cup \{ (Z \backslash z) \rightarrow Y) \}$
                    \If{ $z \in (Z \backslash z)^+_H$}
                        \State $Z \gets Z \backslash z$
                    \EndIf
                \EndIf
            \EndFor
            \If{$X \not = Z$}
                \State $G \gets H$
            \EndIf
        \EndIf
    \EndFor
    \State Remove all duplicated dependencies in G.
    \State \textbf{return} $G$
    \EndFunction
    \hfill \break
    
    \Function{removeRedundantDependencies}{$R$}
    \State $F_m \gets removeExtraneousAttributes(R)$
    \ForAll{ $X \rightarrow Y \in F_m$}
        \While{ $Z \rightarrow Y \in F_m$ exists}
            \State $G \gets F_m \backslash (Z \rightarrow Y)$
            \If {$X \subseteq Y_G^+$}
                \State $F_m \gets G$
            \EndIf
        \EndWhile
    \EndFor
    \State \textbf{return} $F_m$
    \EndFunction
    \hfill \break
    
    \Function{createFullandPartialDependencies}{$R$}
    \State $F_p = []$
    \State $F_f = removeRedundantDependencies(R)$
    \ForAll{$X \rightarrow Y \in F_f$}
        \If{$X \subseteq \{\text{lehetséges kulcsok halmazai}\} \text{ és } Y \not \in K$}
            \State $F_p \gets F_p \cup \{ X \rightarrow Y \}$
            \State $F_f \gets F_f \backslash \{X \rightarrow Y\}$
        \EndIf
        \While{$U \rightarrow V \in F_f \text{ és } U \in Y^+$}
            \State $F_p \gets F_p \cup \{U \rightarrow V\}$
            \State $F_f \gets F_f \backslash \{U \rightarrow V\}$
        \EndWhile
    \EndFor
    \State \textbf{return} ($F_p$,$F_f$) 
    \EndFunction
    \hfill \break
\end{algorithmic}

Elég sok algoritmus, de ezek segítségével meg tudjuk határozni a relációséma teljes és részleges függéseit. Ezután már csak az következik, hogy \nfk-re tudjunk hozni. Előtte egy példán keresztül nézzük meg, hogy mi is történik a függvények által.

\begin{pld}
Legyen $R(\underline{a},\underline{b},c,d,e,f,g,h,i)$ relációséma és vegyük az alábbi függéshalmazt:
\begin{center}
    $\{b,d\} \longrightarrow \{f\}$, 
    $\{b\} \longrightarrow \{d\}$, 
    $\{a,b,d\} \longrightarrow \{f\}$, 
    $\{f\} \longrightarrow \{d\}$, \break
    $\{b,d\} \longrightarrow \{h\}$ ,
    $\{b\} \longrightarrow \{e\}$, 
    $\{a,b,d\} \longrightarrow \{c\}$ ,
    $\{b,d\} \longrightarrow \{g\}$, \break
    $\{b\} \longrightarrow \{i\}$ ,
    $\{b,d\} \longrightarrow \{i\}$.\break
\end{center}

A createAttributePlus algoritmusát a \ref{closedAttribute} tételben található. Az $(RS,LS,LR)$ hármas az elemek vizsgálatából adódik és a példában az alábbi halmazok jönnek így létre:

\begin{center}
    $RS: \{c,e,g,h,i\}$ \break
    $LS: \{a,b\}$\break
    $LR: \{d,f\}$
\end{center}

Ebben a példában is látható, hogy a kulcs megegyezik az $LS$ halmazzal.\par
A következő folyamat a removeExtraneousAttributes, amely eltávolítja a függésekből az implied extraneous attribútumokat. A függvény futtatása után az alábbi függéshalmazt kapjuk vissza:

\begin{center}
    $\{b\} \longrightarrow \{d\}$, 
    $\{f\} \longrightarrow \{d\}$, 
    $\{b\} \longrightarrow \{e\}$, 
    $\{b\} \longrightarrow \{i\}$,\break 
    $\{b\} \longrightarrow \{f\}$, 
    $\{a,b\} \longrightarrow \{f\}$, 
    $\{b\} \longrightarrow \{h\}$, 
    $\{a,b\} \longrightarrow \{c\}$, 
    $\{b\} \longrightarrow \{g\}$.
\end{center}

Vegyük például a $\{b,d\} \longrightarrow \{f\}$-et. Ebben a függésben a $d$ implied extraneous attribútum, mivel $\{b\} \longrightarrow \{d\}$ is teljesül. Emiatt $f$ meghatározásához nincs szükség a $d$ attribútumra, így elhagyható a függésből. \par

Ezután legyen a removeRedundantDependencies, amely azokat a függéseket távolítja el, amelyek nem relevánsak a halmazban. Ebben a példában ezek a következők:

\begin{center}
    $\{f\} \longrightarrow \{d\}$, 
    $\{b\} \longrightarrow \{e\}$, 
    $\{b\} \longrightarrow \{i\}$,
    $\{b\} \longrightarrow \{f\}$, \break 
    $\{b\} \longrightarrow \{h\}$, 
    $\{a,b\} \longrightarrow \{c\}$, 
    $\{b\} \longrightarrow \{g\}$.
\end{center}

Láthatjuk, hogy például a $\{b\} \longrightarrow \{d\}$ függés eltűnt, mivel a  $\{b\} \longrightarrow \{f\}$ és $\{f\} \longrightarrow \{d\}$ is fennáll. Így az első függés elhagyható, mivel már a $\{b\}^+ \subseteq {f}^+$, így törlődik. 

Végül pedig válasszuk szét a createFullandPartialDependencies függvény segítségével a függéseinket. Ez alapján a két halmaz:

\begin{center}
    $F_p = \{ \{f\} \longrightarrow \{d\}, 
    \{b\} \longrightarrow \{e\}, 
    \{b\} \longrightarrow \{i\},
    \{b\} \longrightarrow \{f\}, \break
    \{b\} \longrightarrow \{h\}, 
    \{a,b\} \longrightarrow \{c\}, 
    \{b\} \longrightarrow \{g\} \}$ \break
    $F_f = \{\{a,b\} \longrightarrow \{c\} \}$
\end{center}

Ezek azok alapján lett szétosztva, hogy
\begin{enumerate}
    \item A kezdőfüggés $\{ \text{elsődleges attribútum} \} \longrightarrow \{ \text{nem kulcs}\}$-e,
    \item A többi függés bal oldala visszavezethető-e a kezdőfüggés jobb oldalából.
\end{enumerate}

Így látható is, hogy míg például a $\{b\} \longrightarrow \{e\}$ az $1.$ állítás, addig az $\{f\} \longrightarrow \{d\}$ a $2.$ állítás miatt került az $F_p$-be. Ami még észrevehető, hogy az $F_f$ halmazban lévő függések bal oldalai mindig a teljes kulcsot tartalmazzák.
\end{pld}

Mostmár minden készenáll a \textbf{2NF} bemutatásához:

\begin{algorithmic}
    \Function{2NF}{$R$}
        \State $R2nf \gets []$
        \State $G \gets createFullandPartialDependencies(R)[0]$
        \If{$G == \emptyset$}
            \State \textbf{return} $[R]$
        \EndIf
        \If{$createFullandPartialDependencies(R)[1] \not = \emptyset$}
            \State $X_f \gets createFullandPartialDependencies(R)[1] elements$
        \Else
            \State $X_f \gets A$
        \EndIf
        \ForAll{$X \longrightarrow Y \in G$}
            \If{$X \in K \text{ és } G \not = \emptyset$}
                \State $R(X^+_G).keys \gets X$
                \State $X_f \gets X_f \backslash \{x_i \in X_f \big | x_i \not \in K \text{ és } x_i \in Y_G^+ \}$
                \State $G \gets G \backslash \{ U \rightarrow V \in G \big | U \in Y_G^+ \}$
                \State $R2nf \gets R(X^+_G)$
            \EndIf
        \EndFor
        \State $R2nf \gets R2nf \cup R(X_f)$
        \State $R(X_f).keys \gets X_f$
        \State \textbf{return} $R2nf$
    \EndFunction
\end{algorithmic}

A cikkhez képest itt viszont egy kisebb eltérés található, az az $X_f$ meghatározása. Mivel a négy típusfeladatból három esetben nincs teljes függés, emiatt az $X_f$-be ebben az esetben a teljes attribútumlistát kell elhelyezni. Ez azért fog kelleni, mert ha az $X_f$ nem tartalmaz elemet, abban az esetben minden új reláció üres halmazzal lenne megvalósítva. \par

A cikkben található \textbf{3NF} azért nem lett végül kidolgozva, mivel az algoritmus egyes feladattípusokon nem működőképes. Vegyünk egy példát a zárt feladat típus esetén:

\begin{pld}
Legyen $R(a,b,c,d,e,f,g,h)$ és legyen $F$ az alábbi függéshalmaz:
\begin{center}
    $\{f\} \longrightarrow \{b,d,e,h\},$\break
    $\{b\} \longrightarrow \{f,g\},$\break
    $\{e,h\} \longrightarrow \{d\},$\break
    $\{c,d\} \longrightarrow \{a\}$
\end{center}
Ennek a \textbf{2NF} alakja a következő:
\begin{center}
    $R1(b,d,e,f,g,h)$ \break
    $R2(a,c,f)$
\end{center}
És a cikkben található algoritmus alapján az alábbi relációmodell lenne a \textbf{3NF}:
\begin{center}
    $R1(a,b,d,e,h)$ \break
    $R2(b,f,g)$ \break
    $R3(e,h,d)$ \break
    $R4(c,d,a)$
\end{center}
Viszont elég lenne az $\{e,h\} \longrightarrow \{d\}$ mentén bontani, ami meg is határozza számunkra a \nfh-et:
\begin{center}
    $R1(b,e,f,g,h)$ \break
    $R2(d,e,h)$
    $R3(a,c,f)$
\end{center}
\end{pld}

\section{Összegzés}

Ebben a fejezetben láthatjuk, hogy hogyan lehet általánosságban egy relációs sémát a megfelelő alakra hozni, így ezeket egy függvényhívás segítségével megvalósítani. Látható, hogy a \textbf{2NF} meghatározás nem a mindig a legegyszerűbb meghatározni, és látható, hogy milyen folyamatok történnek egy \nfk-re hozásig. Innentől már csak implementálni kell a moodle-be ezeket a folyamatokat. 